\documentclass[12pt,english,paper=a4,DIV=12,headings=small,numbers=noenddot,parskip=half]{scrartcl}
\usepackage[utf8]{inputenc}
\usepackage[T1]{fontenc}
\usepackage[british]{babel}
\usepackage{amsmath}
\usepackage{amsfonts}
\usepackage{amssymb}
\usepackage{amsthm}
\usepackage{xfrac}
\usepackage{microtype,fixltx2e,ellipsis}
\usepackage{lmodern}
\usepackage{mathrsfs}
\usepackage{enumerate}
\usepackage{charter}


\setkomafont{sectioning}{\bfseries} 

\usepackage[numbers]{natbib}

\theoremstyle{definition}
\newtheorem{defi}{Definition}[section]

\begin{document}

\title{Pairs of Quadrics: Heuristics}
\author{Nils Alex}
\date{\today}

\maketitle

\begin{abstract}
We lift some restrictions that were made in earlier discussions of quadric pairs in order to arrive at a fully (projective-)geometric treatment. Two generic properties of these pairs are proposed: Uniqueness of the centre and symmetry of the quadrics with respect to this centre.
\end{abstract}

\section{The Concept}

As sketched in \ref{Alex:Th12}, for any distinguished point $\mathrm{P}$ in the real plane, the notion of \emph{distance} to this point given by nondegenerate symmetric bilinear forms can be rephrased employing two quadrics. These quadrics $Q_1, Q_2$ have to meet one of the two following requirements (with $N$ denoting the number of lines through $\mathrm{P}$ and $\mathscr{L}_\mathrm{P}$ their entirety):
\begin{enumerate}[i.]
\item{$Q_1 \cap Q_2 = \varnothing$. $\frac{N}{2}$ lines of $\mathscr{L}_\mathrm{P}$ intersect $Q_1$, the other half $Q_2$. Each line is thereby intersected twice by its respective quadric.\label{qp_prop_1}} 
\item{$Q_1 \cap Q_2 = \{ \mathrm{R} \}$. Two lines of $\mathscr{L}_\mathrm{P}$ intersect the quadrics only in $\mathrm{R}$. For the remaining $N-2$ lines and the by $\mathrm{R}$ reduced quadrics, \ref{qp_prop_1}. holds.}
\end{enumerate}

This choice of quadrics serves twofold: First, a quadric naturally assigns to each point its respective line, namely its \emph{polar}, and vice versa. After choosing an ``infinity line'' $\ell_\infty$ in a projective plane $\mathscr{P}$ one can establish an addition on $\mathscr{P}\setminus\ell_\infty$ as well as a multiplication. This has been thoroughly studied in \cite{Beutelspacher}. Whether both quadrics yield the same polar is an important question to be addressed later on. 

Second, every line in $\mathscr{P}\setminus\ell_\infty$ can be thought of as a field with addition and multiplication being the ones just mentioned. Therefore, a certain choice

\section{Definiton of Pair Quadrics}

After a more general introduction of the concept, the discussion of pair quadrics in \cite{Alex:Th12} was restricted by further assumptions, namely that (in an appropriate coordinatisation of the plane)
\begin{enumerate}[i.]
\item{both quadrics have the \emph{centre} $\mathrm{P}=\langle1,0,0\rangle$,}
\item{the polar of $\mathrm{P}$ with respect to both quadrics is the line $\ell_\infty=\lbrack1,0,0\rbrack$, and}
\item{both quadrics possess \emph{point symmetry} with respect to $\mathrm{P}$, i.e.\ for each $\mathrm{S}=\langle1,x_1,x_2\rangle$ in $Q$, the point $\mathrm{-S}=\langle1,-x_1,-x_2\rangle$ is also contained therein.\label{symmetry}}
\end{enumerate}

In a finite projective plane of order $n>3$ a point $\mathrm{P}$ is called a \emph{centre} with respect to a quadric pair $(Q_1,Q_2)$ if either
\begin{enumerate}[{i}a]
\item{$Q_1$ intersects exactly $\frac{N}{2}$ of the lines incident with $\mathrm{P}$, $Q_2$ the other half. }
\end{enumerate}

While whe want to maintain (\ref{symmetry}) (but rephrase it generically (see \ref{sect:symmetry}), the first two assumptions are very crucial: We equipped the projective plane with a distinguished line

\begin{defi}
bla
\end{defi}

\end{document}
