\documentclass[12pt,english,paper=a4,DIV=12,headings=small,numbers=noenddot,parskip=half]{scrartcl}
\usepackage[utf8]{inputenc}
\usepackage[T1]{fontenc}
\usepackage[british]{babel}
\usepackage{amsmath}
\usepackage{amsfonts}
\usepackage{amssymb}
\usepackage{amsthm}
\usepackage{microtype,fixltx2e,ellipsis}
\usepackage{lmodern}
\usepackage{mathrsfs}
\usepackage{enumerate}

\setkomafont{sectioning}{\bfseries} 

\usepackage[numbers]{natbib}

\theoremstyle{definition}
\newtheorem{defi}{Definition}[section]

\begin{document}

\title{Pairs of Quadrics: Heuristics}
\author{Nils Alex}
\date{\today}

\maketitle

\begin{abstract}
We lift some restrictions that were made in earlier discussions of quadric pairs in order to arrive at a fully (projective-)geometric treatment. Two generic properties of these pairs are proposed: Uniqueness of the centre and symmetry of the quadrics with respect to this centre.
\end{abstract}

\section{The Concept}

As sketched in \ref{Alex:Th12}, the notion of \emph{distance} in the real plane given by nondegenerate symmetric bilinear forms can be rephrased using two quadrics

\section{Definiton of Pair Quadrics}

After a more general introduction of the concept, the discussion of pair quadrics in \cite{Alex:Th12} was restricted by further assumptions, namely that (in an appropriate coordinatisation of the plane)
\begin{enumerate}[i]
\item{both quadrics have the \emph{centre} $\mathrm{P}=\langle1,0,0\rangle$,}
\item{the polar of $\mathrm{P}$ with respect to both quadrics is the line $\ell_\infty=\lbrack1,0,0\rbrack$, and}
\item{both quadrics possess \emph{point symmetry} with respect to $\mathrm{P}$, i.e.\ for each $\mathrm{S}=\langle1,x_1,x_2\rangle$ in $Q$, the point $\mathrm{-S}=\langle1,-x_1,-x_2\rangle$ is also contained therein.\label{symmetry}}
\end{enumerate}

In a finite projective plane of order $n>3$ a point $\mathrm{P}$ is called a \emph{centre} with respect to a quadric pair $(Q_1,Q_2)$ if either
\begin{enumerate}[{i}a]
\item{$Q_1$ intersects exactly $\frac{N}{2}$ of the lines incident with $\mathrm{P}$, $Q_2$ the other half. }
\end{enumerate}

While whe want to maintain (\ref{symmetry}) (but rephrase it generically (see \ref{sect:symmetry}), the first two assumptions are very crucial: We equipped the projective plane with a distinguished line

\begin{defi}
bla
\end{defi}

\end{document}